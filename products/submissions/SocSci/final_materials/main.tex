% Options for packages loaded elsewhere
\PassOptionsToPackage{unicode}{hyperref}
\PassOptionsToPackage{hyphens}{url}
\PassOptionsToPackage{dvipsnames,svgnames,x11names}{xcolor}
%
\documentclass[
  12pt,
  letterpaper,
]{article}

\usepackage{amsmath,amssymb}
\usepackage{lmodern}
\usepackage{setspace}
\usepackage{iftex}
\ifPDFTeX
  \usepackage[T1]{fontenc}
  \usepackage[utf8]{inputenc}
  \usepackage{textcomp} % provide euro and other symbols
\else % if luatex or xetex
  \usepackage{unicode-math}
  \defaultfontfeatures{Scale=MatchLowercase}
  \defaultfontfeatures[\rmfamily]{Ligatures=TeX,Scale=1}
\fi
% Use upquote if available, for straight quotes in verbatim environments
\IfFileExists{upquote.sty}{\usepackage{upquote}}{}
\IfFileExists{microtype.sty}{% use microtype if available
  \usepackage[]{microtype}
  \UseMicrotypeSet[protrusion]{basicmath} % disable protrusion for tt fonts
}{}
\makeatletter
\@ifundefined{KOMAClassName}{% if non-KOMA class
  \IfFileExists{parskip.sty}{%
    \usepackage{parskip}
  }{% else
    \setlength{\parindent}{0pt}
    \setlength{\parskip}{6pt plus 2pt minus 1pt}}
}{% if KOMA class
  \KOMAoptions{parskip=half}}
\makeatother
\usepackage{xcolor}
\usepackage[margin=1in]{geometry}
\setlength{\emergencystretch}{3em} % prevent overfull lines
\setcounter{secnumdepth}{-\maxdimen} % remove section numbering
% Make \paragraph and \subparagraph free-standing
\ifx\paragraph\undefined\else
  \let\oldparagraph\paragraph
  \renewcommand{\paragraph}[1]{\oldparagraph{#1}\mbox{}}
\fi
\ifx\subparagraph\undefined\else
  \let\oldsubparagraph\subparagraph
  \renewcommand{\subparagraph}[1]{\oldsubparagraph{#1}\mbox{}}
\fi


\providecommand{\tightlist}{%
  \setlength{\itemsep}{0pt}\setlength{\parskip}{0pt}}\usepackage{longtable,booktabs,array}
\usepackage{calc} % for calculating minipage widths
% Correct order of tables after \paragraph or \subparagraph
\usepackage{etoolbox}
\makeatletter
\patchcmd\longtable{\par}{\if@noskipsec\mbox{}\fi\par}{}{}
\makeatother
% Allow footnotes in longtable head/foot
\IfFileExists{footnotehyper.sty}{\usepackage{footnotehyper}}{\usepackage{footnote}}
\makesavenoteenv{longtable}
\usepackage{graphicx}
\makeatletter
\def\maxwidth{\ifdim\Gin@nat@width>\linewidth\linewidth\else\Gin@nat@width\fi}
\def\maxheight{\ifdim\Gin@nat@height>\textheight\textheight\else\Gin@nat@height\fi}
\makeatother
% Scale images if necessary, so that they will not overflow the page
% margins by default, and it is still possible to overwrite the defaults
% using explicit options in \includegraphics[width, height, ...]{}
\setkeys{Gin}{width=\maxwidth,height=\maxheight,keepaspectratio}
% Set default figure placement to htbp
\makeatletter
\def\fps@figure{htbp}
\makeatother

% -----------------------
% CUSTOM PREAMBLE STUFF
% -----------------------

\usepackage{orcidlink}
\definecolor{mypink}{RGB}{219, 48, 122}
\usepackage{endnotes}
\usepackage[nolists]{endfloat}

% ---------------------------
% END CUSTOM PREAMBLE STUFF
% ---------------------------
\usepackage{booktabs}
\usepackage{longtable}
\usepackage{array}
\usepackage{multirow}
\usepackage{wrapfig}
\usepackage{float}
\usepackage{colortbl}
\usepackage{pdflscape}
\usepackage{tabu}
\usepackage{threeparttable}
\usepackage{threeparttablex}
\usepackage[normalem]{ulem}
\usepackage{makecell}
\usepackage{xcolor}
\makeatletter
\makeatother
\makeatletter
\makeatother
\makeatletter
\@ifpackageloaded{caption}{}{\usepackage{caption}}
\AtBeginDocument{%
\ifdefined\contentsname
  \renewcommand*\contentsname{Table of contents}
\else
  \newcommand\contentsname{Table of contents}
\fi
\ifdefined\listfigurename
  \renewcommand*\listfigurename{List of Figures}
\else
  \newcommand\listfigurename{List of Figures}
\fi
\ifdefined\listtablename
  \renewcommand*\listtablename{List of Tables}
\else
  \newcommand\listtablename{List of Tables}
\fi
\ifdefined\figurename
  \renewcommand*\figurename{Figure}
\else
  \newcommand\figurename{Figure}
\fi
\ifdefined\tablename
  \renewcommand*\tablename{Table}
\else
  \newcommand\tablename{Table}
\fi
}
\@ifpackageloaded{float}{}{\usepackage{float}}
\floatstyle{ruled}
\@ifundefined{c@chapter}{\newfloat{codelisting}{h}{lop}}{\newfloat{codelisting}{h}{lop}[chapter]}
\floatname{codelisting}{Listing}
\newcommand*\listoflistings{\listof{codelisting}{List of Listings}}
\makeatother
\makeatletter
\@ifpackageloaded{caption}{}{\usepackage{caption}}
\@ifpackageloaded{subcaption}{}{\usepackage{subcaption}}
\makeatother
\makeatletter
\@ifpackageloaded{tcolorbox}{}{\usepackage[many]{tcolorbox}}
\makeatother
\makeatletter
\@ifundefined{shadecolor}{\definecolor{shadecolor}{rgb}{.97, .97, .97}}
\makeatother
\makeatletter
\makeatother
\ifLuaTeX
  \usepackage{selnolig}  % disable illegal ligatures
\fi
\usepackage[style=apa,]{biblatex}
\addbibresource{project.bib}
\IfFileExists{bookmark.sty}{\usepackage{bookmark}}{\usepackage{hyperref}}
\IfFileExists{xurl.sty}{\usepackage{xurl}}{} % add URL line breaks if available
\urlstyle{same} % disable monospaced font for URLs
\hypersetup{
  pdftitle={Differences in the Risk of Grade Retention for Biracial and Monoracial Students in the US, 2010-2019},
  pdfauthor={Aaron Gullickson},
  colorlinks=true,
  linkcolor={blue},
  filecolor={Maroon},
  citecolor={Blue},
  urlcolor={Blue},
  pdfcreator={LaTeX via pandoc}}

\title{Differences in the Risk of Grade Retention for Biracial and
Monoracial Students in the US, 2010-2019\thanks{Supplementary materials
provided with this article include full model results upon which figures
are based as well as any sensitivity analysis described herein. All code
and data for this project are available at
\url{https://osf.io/4fevt/?view_only=4abc6d86595c4313a8d4792471e9bc0d}.}}
\author{
Aaron Gullickson~\orcidlink{0000-0001-7237-8131}\\University of
Oregon, Sociology\\\href{mailto:aarong@uoregon.edu}{aarong@uoregon.edu}}
\date{}
\begin{document}


\maketitle
\begin{abstract}
Understanding how outcomes for biracial individuals compare to their
monoracial peers is critical for understanding how patterns of racial
inequality in the contemporary United States might be shifting. Yet, we
know very little about the life chances of biracial individuals due to
limitations in most available data sources. In this article, I utilize
American Community Survey data from 2010-2019 to examine the risk of
being clearly behind expected grade among biracial and monoracial K-12
students, helping to fill a gap in our understanding. With large sample
sizes for most biracial groups, I am able to estimate grade retention
risk for biracial students with enough precision to differentiate even
modest differences in risk relative to monoracial groups. The results
indicate that for most biracial groups, biracial students have risk
similar to their monoracial constituent group with lower risk. Although
biracial students tend to have favorable family resource
characteristics, controlling for these characteristics does little to
change the overall placement of their outcomes.
\end{abstract}

\ifdefined\Shaded\renewenvironment{Shaded}{\begin{tcolorbox}[frame hidden, boxrule=0pt, borderline west={3pt}{0pt}{shadecolor}, enhanced, interior hidden, breakable, sharp corners]}{\end{tcolorbox}}\fi

\setstretch{1}
\hypertarget{introduction}{%
\section{Introduction}\label{introduction}}

The experiences of mixed race individuals will play an important role in
the future of race in the US. A rise in interracial unions since the
1960s has resulted in a ``biracial baby boom'' of individuals whose
parents cross racial lines \autocite{root_racially_1992} and this
demographic surge may have transformative effects on racial boundaries
\autocite{hochschild_creating_2012,alba_great_2020}. While racial mixing
is nothing new to US history
\autocite{morning_who_2000,gullickson_choosing_2011}, this biracial boom
is notable for its size and has occurred in an environment in which
historical norms of racial classification may be eroding.

The transformative potential of this biracial baby boom is contingent on
both how these biracial individuals identify their own race and how they
are sorted into a prevailing system of racial inequality. While
extensive scholarship documents the identification and classification
decisions made by or on behalf of multiracial individuals
\autocite{xie_racial_1997,qian_options_2004,roth_end_2005a,brunsma_interracial_2005,herman_forced_2004,holloway_place_2009,bratter_mother_2009,davenport_role_2016,liebler_boundaries_2016},
only a smattering of research has documented how the life chances of
biracial individuals compare to their monoracial peers
\autocite{kao_racial_1999,campbell_multiracial_2009,herman_blackwhiteother_2009}.
In this article, I expand on this work by examining the risk of grade
retention for K-12 students whose parents identify with different races.

By examining this outcome, I am able to overcome two important
limitations that have hampered prior work on the life chances of
biracial individuals. First, using self identification of biracial
respondents to analyze outcomes is problematic. Prior research indicates
that individuals whose parents belong to different races do not
consistently identify with both races
\autocite{doyle_are_2007,liebler_america_2017,gullickson_racial_2019}.
Furthermore, social status and social mobility can predict racial
identification generally
\autocite{saperstein_racial_2012,saperstein_mulatto_2013} and social
status is related to the identification choices of biracial individuals
specifically \autocite{roth_end_2005a,davenport_role_2016}. Thus,
differences in outcomes among biracial individuals may in part produce
differences in identification, rather than the other way around.

This problem of reverse causation can be eliminated when researchers
have access to the racial identification of a respondent's biological
parents. Even when such information is available, however, researchers
run into a second limitation of sample size. The samples of biracial
respondents in many data sources are too small to make reliable
statistical inferences. This limitation is particularly severe for
biracial individuals with two non-white parents.

Biracial children who reside with both biological parents can be
identified in large scale data sources such as the Census. Researchers
have previously utilized this feature to examine the classification
choices made on behalf of these children
\autocite{xie_racial_1997,roth_end_2005a}. Analyzing outcomes for such
children is more difficult, for the simple reason that they do not have
many measurable outcomes yet. However, grade retention is measurable and
important even for young children. Being held back in school is a
relatively common occurrence that has consequences for later educational
transitions \autocite{jimerson_metaanalysis_2001} and is
disproportionately applied by race \autocite{warren_patterns_2014}.
Grade retention is one of the earliest stratifiers of life chances that
an individual will experience.

In this article, I use data from the American Community Survey to
examine the likelihood of being held back in school for biracial
children in comparison to their monoracial peers. Pooling data from 2010
to 2019, I am able to draw a large enough sample to make reliable
statistical inferences for most biracial groups, including those with
two non-white parents. The results provide a window into the placement
of biracial children within an existing system of racial inequality. In
contrast to prevailing hypotheses that predict either a pattern of
hypodescent or an in-between position for biracial individuals, I find a
consistent pattern in which the risk of grade retention is similar to
that of the lower risk constituent monoracial group. This result holds
even after adjusting for differences in resources across respondents,
despite the fact that most biracial groups are positively selected by
parental resources relative to their monoracial group.

\hypertarget{where-do-biracials-fit}{%
\section{Where do Biracials Fit?}\label{where-do-biracials-fit}}

How might we expect the outcomes for biracial individuals to differ from
their monoracial peers? In every case, the relevant comparison is how a
biracial person fares relative to the two constituent groups with which
their parents identify. For example, in the case of an individual with
one Black and one White parent, we would compare the average outcomes
for such individuals to the average outcomes for individuals with two
White and two Black parents, respectively. The outcome for the
Black/White group may be similar to one of the monoracial groups, it may
be somewhere in between the outcomes for the two monoracial groups, or
it may be higher or lower than either of the monoracial groups.

I distinguish three mechanisms that help us understand where biracial
individuals might be positioned among these possibilities. First, I
consider arguments regarding racial discrimination at both an individual
and systemic level. Second, I consider arguments regarding different
distributions of resources across racial groups. Third, I consider
arguments regarding how the nature of hybridity itself may affect
outcomes.

\hypertarget{discrimination-and-classification}{%
\subsection{Discrimination and
Classification}\label{discrimination-and-classification}}

Discrimination and bias by race play an important role in overall levels
of racial inequality \autocite{quillian_new_2006,pager_sociology_2008}.
Discrimination need not be consciously recognized by actors, but may
instead reflect cognitive biases and schema that individuals unknowingly
rely upon and reinforce in social interactions
\autocite{quillian_new_2006}. Discrimination can also be systemic in
nature when biases are built into and reinforced by organizations and
institutional systems \autocite{reskin_race_2012}. In our current
``colorblind'' regime \autocite{bonilla-silva_racism_2006}, such systems
may be formally race neutral but produce disparate impacts by race
\autocite{pager_sociology_2008}.

The experiences of biracial individuals within such a system of
discrimination depend almost entirely on how they are classified by
others. To the extent that biracial individuals are classified as more
like one of their monoracial constituent groups than the other, they
will receive the same treatment and reap similar rewards or suffer
similar penalties.

The strongest historical precedent for understanding such classification
in the US is the ``one-drop rule'' norm governing classification as
Black. According to the one-drop rule, individuals of known Black
ancestry are considered exclusively Black, regardless of other
ancestries \autocite{davis_who_1991}. \textcite{davis_who_1991}
generalizes this practice into the concept of \emph{hypodescent} in
which individuals with mixed racial ancestry will be classified as
members of the lower status group. A parallel pattern of
\emph{hyperdescent} implies that individuals of mixed racial ancestry
will be classified as members of the higher status group. This pattern
of hyperdescent has historically governed the experiences of individuals
with American Indian ancestry
\autocite{snipp_american_1989,wolfe_land_2001}.

Although the hypodescent/hyperdescent paradigm is often used to frame
the experiences of multiracial individuals, its generalizability and
contemporary relevance is questionable. Hypodescent and hyperdescent are
observed in two specific historical cases involving Black ancestry and
American Indian ancestry, respectively. As
\textcite{iverson_regimes_2022} note, these two cases could just as
parsimoniously be explained by a ``dominance'' model which ranks
ancestries by their tendencies to be supercessive or recessive in
determining identification and classification.

The development of particular norms may also reflect the historical
regime in which they were developed
\autocite{gullickson_choosing_2011,iverson_regimes_2022}. In this case,
neither the hypodescent/hyperdescent or dominance paradigm may help us
to fully understand the experiences of those of Latino and Asian
ancestry due to more recent processes of racialization. Furthermore, the
historical norms for all groups may be transforming in the post-Civil
Rights era.

If outcomes for biracial individuals all closer to one of their
constituent groups, then we may have stronger evidence about such norms
of classification. However, what if those outcomes fall in-between the
two groups? Processes of classification and discrimination could still
produce an in-between status for several reasons.

First, racial ambiguity in appearance may lead to more inconsistent
classification into single race categories across observers. Even if
norms consistently indicate a certain type of classification, correct
identification by observers is a probabilistic phenomenon and likely to
be more heterogeneous for biracial individuals. As a result, biracial
individuals will experience a lower sum total of discrimination and bias
affecting the lower status group in the case of hypodescent and will not
reap the same rewards as a member of the higher status group in the case
of hyperdescent.

Second, biracial people's experience of discrimination may be
``softened'' because a more ambiguous physical appearance provokes less
racial antipathy. The US already has a long history of this kind of
softening. Because of the one-drop rule, Black individuals vary
substantially in skin tone and those individuals with a lighter skin
tone have better outcomes
\autocite{keith_skin_1991,hughes_significance_1990,monk_skin_2014} and
are viewed more favorably by both Blacks and Whites
\autocite{maddox_cognitive_2002a}. Thus, light skin moderates the stigma
of Blackness even among monoracial individuals.

Third, biracial individuals may be classified into an ambiguous
``middleman'' minority position. This kind of middle tier status for
individuals of mixed race is common in Latin America
\autocite{telles_race_2009} and \textcite{bonilla-silva_biracial_2004}
has argued that such a system may be emerging in the contemporary US as
well. In this case, many biracial groups may occupy a nebulous and
ill-defined middleman or buffer class position in-between whites and
darker-skinned non-whites, helping to cement a new kind of racial
hierarchy.

\hypertarget{differential-resources}{%
\subsection{Differential Resources}\label{differential-resources}}

Racial inequality is at least partly explained by different
distributions of material, cultural, and social resources across racial
groups \autocite{conley_being_1999}. On average, socioeconomic
background (e.g.~parental income, wealth, and education) differs by race
and such socioeconomic background is consequential for outcomes. Other
less materially defined resources also vary across groups such as
cultural styles, social networks, and neighborhood quality. These
differences in background resources reflect the accumulation of
historical processes of discrimination and, when combined with
contemporary processes of discrimination, magnify the overall level of
racial inequality we observe.

How do differences in the distribution of resources affect biracial
individuals? Naively, we might expect biracial individuals to inherit
parental resources roughly halfway between their two constituent groups,
on average \autocite{chew_american_1989}. Prior research looking at
biracial children in the Census data has found some evidence to support
this argument \autocite{chew_american_1989}. In such a case, the
resources of biracial individuals would predict outcomes in-between the
two constituent groups, barring other mechanisms.

However, the actual resource position of biracial individuals is likely
far more complicated because people are not randomly selected into
interracial unions. The relationship between this selection process and
a biracial child's outcomes depends on the strength and direction of
that selection on each partner, its consistency across different types
of unions, and how strongly the selected characteristic predicts
outcomes in the next generation.

Status exchange theory predicts that given an unequal racial hierarchy,
individuals from a lower status racial group will be positively selected
on socioeconomic characteristics into intermarriage with a higher status
group, while individuals from a higher status racial group will be
similarly negatively selected into the same marriage
\autocite{davis_intermarriage_1941,merton_intermarriage_1941,fu_racial_2001}.
Prior research has shown substantial evidence of this pattern in the
case of Black male/White female marriages in the US, but weaker evidence
for other types of unions
\autocite{gullickson_education_2006,fu_racial_2001,kalmijn_educational_2010,hou_interracial_2011}.
Even in cases where status exchange is present, its net effect on
resources is complex because status exchange positively selects one
partner and negatively selects the other partner. Therefore the two
effects may somewhat cancel each other out in terms of the overall
distribution of resources across parents.

Regardless of the underlying complexity of this selection process,
delineating its role as opposed to other mechanisms is key to
understanding the overall process that sorts biracial individuals within
the racial hierarchy. Although this approach is sometimes characterized
as pitting race versus class, that is not the case here. Instead, this
delineation allows us to understand how race affects the outcomes of
biracial individuals in a multigenerational process.

\hypertarget{hybridity}{%
\subsection{Hybridity}\label{hybridity}}

The experience of hybridity itself may also positively or negatively
affect outcomes for biracial individuals. According to ``marginal man''
theory, biracial individuals will find it difficult to fit in due to the
dissonance of an identity that crosses strong and salient racial
boundaries \autocite{park_human_1928,stonequist_problem_1935}. This
liminal identity conflict will result in negative mental health
experiences that could feed into other negative outcomes as well.
Evidence for such negative mental health experiences, however, has been
mixed
\autocite{udry_health_2003,campbell_what_2006,cheng_multiracial_2009,bratter_does_2011}.

Marginal man theorists viewed culture in monolithic and static terms and
thus to be trapped in between two cultures was seen as a difficult
experience. Contemporary views of culture instead emphasize the
flexible, fragmented, and fluid use of cultural bits that comprise a
``toolkit'' \autocite{swidler_culture_1986,dimaggio_culture_1997}. From
this perspective, biracial individuals may actually benefit from an
enlarged cultural toolkit that allows them to navigate a variety of
diverse social contexts \autocite{shih_multiple_2019}.

The complexity of the underlying mechanisms described above prevents
simple and straightforward hypotheses regarding the position of biracial
individuals. Different mechanisms may produce the same observed outcome
and the weight of different mechanisms can also vary by groups as these
groups may be governed by different regimes of classification. However,
comparing outcomes across biracial groups may help us better understand
the likely mechanisms involved and to disregard others as implausible.
Accounting for the effect of parental resources on outcomes to the best
of our ability is also key to reaching any conclusions about the
placement of biracial individuals within the racial hierarchy.

\hypertarget{prior-work}{%
\section{Prior Work}\label{prior-work}}

Prior work on the outcomes for biracial individuals has been hampered by
several methodological issues. With good reason, researchers have been
reluctant to use data in which multiracial or biracial status is
determined solely by the self-identification of respondents. We know
that far fewer people identify as biracial than could feasibly do so
based on their parents' races \autocite{morning_generational_2018} and
that various selection effects operate on the decision to identify as
biracial among those who could
\autocite{roth_end_2005a,davenport_role_2016}. Furthermore, research on
changes in racial reporting over time suggests that other status markers
like education and income may influence how people identify themselves
\autocite{saperstein_racial_2012,saperstein_mulatto_2013}. For example,
a person with one Black and one White parent who is routinely treated by
others as Black and discriminated against as Black may be more likely to
identify themselves as exclusively Black than a similar person of Black
and White ancestry who experiences less consistent discrimination. Thus,
any analysis of outcomes for biracial individuals based on
self-identification must deal with the potential for reverse causation
in the observed differences between biracial individuals and their
constituent groups \autocite{bratter_multiracial_2018}.

Instead of self-identification, researchers have generally relied upon
survey data in which the racial identification of a respondent's
biological parents are provided. \textcite{campbell_multiracial_2009}
used data from the National Longitudinal Survey of Adolescent to Adult
Health (Add Health) to examine high school GPA, advanced placement in
math, and four-year college enrollment among biracial and monoracial
individuals identified by co-resident biological parents' races.
\textcite{herman_blackwhiteother_2009} examined differences in GPA using
data on high school students in California and Wisconsin in which the
students reported their parents' races. Both of these studies found some
support for an ``in-between'' status for many biracial groups, but are
hampered by a lack of statistical power due to small sample sizes.
\textcite{kao_racial_1999} used data from the National Educational
Longitudinal Study of 1988 to examine mathematics achievement scores and
GPA for biracial students with Black and Asian parentage and finds that
biracial Black students have outcomes similar to monoracial Black
students, while biracial Asian students have outcomes more similar to
monoracial White students. This study, however, is limited by the fact
that biracial students are identified by a discrepancy between their
reported race and one parent's reported race, rather than by both
biological parents' races.

Results in prior work tend to be inconclusive, due to the sample size
for most biracial groups. An alternative approach is to use large scale
Census or American Community Survey (ACS) data to identify biracial
children by the race of the biological parents in their household. This
approach can generate much larger samples of biracial individuals.
However, researchers then have a limited range of outcomes to explore
because biracial individuals identified by this method are by definition
children. This approach has been used previously to examine the
likelihood of living in poverty
\autocite{bratter_multiracial_2013,bratter_poverty_2013} and residential
segregation \autocite{ellis_agents_2012}. These studies provide valuable
information on biracial childrens' lived experience, but do not directly
measure their own outcomes. However, one important educational outcome
is partially identifiable for children in Census and ACS data - grade
retention.

Grade retention is the practice of having students repeat a grade due to
poor academic performance. Prior work on grade retention has failed to
show a positive effect of grade retention on later educational outcomes,
and has instead found evidence of ``scarring'' effects that lead to
later negative educational outcomes, such as a higher dropout risk
\autocite{jimerson_metaanalysis_2001,stearns_staying_2007,andrew_scarring_2014,hughes_effect_2017}.
The risk of grade retention also varies substantially by race and class
background, with Black and Latino students at particularly high risk of
grade retention relative to other groups
\autocite{warren_patterns_2014}.

Because grade retention often happens very early in the K-12 system, it
is one of the earliest stratification mechanisms that individuals
encounter. For these same reasons, it provides an important window into
understanding the placement of biracial children within a racially
stratified educational environment. In the remainder of this article, I
examine this placement using data from the American Community Survey.

\hypertarget{data-and-methods}{%
\section{Data and Methods}\label{data-and-methods}}

Data for this analysis come from the American Community Survey (ACS), an
annual 1-in-100 survey of the US population, conducted by the Census
Bureau. To increase sample size for smaller populations of biracial
respondents, I pool ACS data for a full decade from 2010-2019. All data
were extracted from the Minnesota Population Center's Integrated Public
Use Microdata Series \autocite{ruggles_ipums_2020}.

Ideally, I would restrict the sample to children living in a household
with two biological parents. However, while the ACS distinguishes
between biological, adopted, and stepchildren, it only records the
direct family relationship between the head of household and other
members of the household. Respondents can therefore be identified as
biological children of another member of the household only if that
member is the head of household or the biological child of the head of
household. Determining whether the partner/spouse of that biological
parent is the other biological parent is more difficult.

Prior work has used a variety of additional restrictions to limit the
analysis to children who are more likely to be the biological children
of both parents \autocite{saenz_persistence_1995,xie_racial_1997}.
Consistent with this prior work, I restrict cases in two ways. First, I
restrict the sample to cases where both parents were of a reasonable age
at the birth of the child (aged 15-44 for mothers and aged 15-60 for
fathers). Second, I restrict the sample to those cases where the
reported race and Hispanicity of the child is inclusive of at least one
of the parents' races.

The final analytical sample consists of children aged 5-20 who are
currently enrolled in the K-12 school system and live in a household
with two parents who are both likely to be biological parents. This
necessary sample restriction will bias the family structure of the
analytical sample relative to the total population of K-12
schoolchildren. Children living in two-parent nuclear households have
greater educational success than other children as a result of material
and social resources \autocite{biblarz_family_1999}. Therefore, this
sample restriction will bias overall estimates of grade retention
downward. However, this bias is less problematic for the goal of this
study, which is to estimate differences between groups. The sample
restriction effectively eliminates differences in family structure
across racial groups that may account for some observed differences in
grade retention. However, the remaining differences observed here are
likely to hold for the full population in the absence of strong racial
differences in the effect of family structure on grade retention.

The race of children is calculated by a cross-tabulation of parents'
races. To produce reasonably parsimonious categories, I collapse each
parent's race into the categories of White, Black, Asian, Latino, and
Indigenous. These categories form the ``ethnoracial pentagon'' that is
commonly used in popular practice and government tabulation to identify
race in the US \autocite{hollinger_postethnic_1995}. Parents are
identified as Latino based on their response to the Hispanicity
question, regardless of their response to the race question. Parents are
identified as Indigenous if they responded as either American
Indian/Alaska Native or as Pacific Islander. Because the goal of my
analysis is to examine specifically the outcomes of ``first-generation''
biracial children, I exclude cases where at least one parent identified
with multiple races. The cross-tabulation of parents' races leads to ten
distinct biracial categories. The total sample size for each of these
biracial categories is shown in Table~\ref{tbl-descriptive}, along with
the sample size of the monoracial comparison groups.

\hypertarget{tbl-descriptive}{}
\begin{table}
\caption{\label{tbl-descriptive}Descriptive survey-weighted statistics by race. Shading indicates
biracial group. Groups ordered by sample size. }\tabularnewline

\centering
\begin{tabular}[t]{lrr}
\toprule
Race & Sample size & Clearly behind expected grade\\
\midrule
White & 1,703,079 & 3.51\%\\
Latino & 386,718 & 5.11\%\\
Asian & 146,479 & 2.56\%\\
Black & 130,849 & 5.45\%\\
\cellcolor[HTML]{D3D3D3}{White/Latino} & \cellcolor[HTML]{D3D3D3}{120,369} & \cellcolor[HTML]{D3D3D3}{3.13\%}\\
\cellcolor[HTML]{D3D3D3}{White/Asian} & \cellcolor[HTML]{D3D3D3}{38,014} & \cellcolor[HTML]{D3D3D3}{1.82\%}\\
\cellcolor[HTML]{D3D3D3}{Black/White} & \cellcolor[HTML]{D3D3D3}{26,829} & \cellcolor[HTML]{D3D3D3}{3.86\%}\\
Indigenous & 16,224 & 7.86\%\\
\cellcolor[HTML]{D3D3D3}{White/Indigenous} & \cellcolor[HTML]{D3D3D3}{13,689} & \cellcolor[HTML]{D3D3D3}{5.25\%}\\
\cellcolor[HTML]{D3D3D3}{Black/Latino} & \cellcolor[HTML]{D3D3D3}{9,578} & \cellcolor[HTML]{D3D3D3}{3.93\%}\\
\cellcolor[HTML]{D3D3D3}{Latino/Asian} & \cellcolor[HTML]{D3D3D3}{5,570} & \cellcolor[HTML]{D3D3D3}{2.43\%}\\
\cellcolor[HTML]{D3D3D3}{Indigenous/Latino} & \cellcolor[HTML]{D3D3D3}{2,627} & \cellcolor[HTML]{D3D3D3}{5.25\%}\\
\cellcolor[HTML]{D3D3D3}{Black/Asian} & \cellcolor[HTML]{D3D3D3}{2,084} & \cellcolor[HTML]{D3D3D3}{2.98\%}\\
\cellcolor[HTML]{D3D3D3}{Black/Indigenous} & \cellcolor[HTML]{D3D3D3}{701} & \cellcolor[HTML]{D3D3D3}{5.96\%}\\
\cellcolor[HTML]{D3D3D3}{Indigenous/Asian} & \cellcolor[HTML]{D3D3D3}{521} & \cellcolor[HTML]{D3D3D3}{2.60\%}\\
\bottomrule
\end{tabular}
\end{table}

Grade retention is difficult to measure accurately. In most data
sources, researchers lack specific reports of grade retention and
instead infer grade retention from a discrepancy between a student's
reported age and grade. Prior studies have used the concept of a student
being behind modal grade
\autocite{bianchi_children_1984,frederick_have_2008}. The most notable
limitation of this approach is the ambiguity created by the fact that at
any given age, a student may reasonably be in two modal grades. Without
detailed information on birthdates and survey timing, the correct modal
grade for most students cannot be identified. This issue has been
somewhat alleviated in prior work by the use of the October supplement
to the Current Population Survey which is close to the beginning of the
school year.

The ACS data do not provide the ability to identify or limit survey
timing, so I instead use a related measure of whether a given student is
clearly behind expected grade (CBEG). A student is considered CBEG if
their age is higher than either of the expected ages for a student of
that grade. This measure will underestimate overall grade retention
because it will miss students who have been retained but not yet had a
birthday in their current grade that would place them CBEG. However, the
goal of this study is not to estimate grade retention accurately, but
rather to understand racial differences in grade retention. Because this
bias is largely a function of when students have birthdays, it should be
more or less random with regard to sociodemographic characteristics.
This approach is equivalent to the method used by
\textcite{rosenfeld_nontraditional_2010} on 2000 Census data, but more
detailed information on current grade in recent ACS data allows for more
precise estimation. Table~\ref{tbl-descriptive} shows the percent of
each monoracial and biracial group that is CBEG.

Using grade-age comparisons to infer grade retention may inadvertently
capture cases of ``academic redshirting'' in which parents intentionally
delay their child's kindergarten enrollment by a year.
\textcite{frederick_have_2008} suggest that academic redshirting is in
many cases be a form of preemptive retention for children with
developmental delays and thus the error induced by these false positive
cases may be minimal. However, \textcite{bassok_academic_2013} have
shown that academic redshirting is more common among white and SES
privileged parents, suggesting that the demographic covariates of
academic redshirting may operate in the opposite direction of grade
retention. Regardless, academic redshirting should be of minimal concern
for the measure of CBEG used here. Academic redshirting most frequently
occurs for students whose birthdays fall close to the cutoff period for
enrollment \autocite{graue_redshirting_2000,bassok_academic_2013}.
Therefore, academic redshirts spend most of a given school year at the
older, but correct, modal age for their grade and most of these students
will not be identified as CBEG.\footnote{Analysis of the probability of
  being CBEG by race and grade, shown in the supplementary materials,
  indicates that Black and Latino children in kindergarten are less
  likely to be CBEG than White children, which may be a consequence of
  redshirting. However, these differences quickly reverse direction by
  first grade. As a sensitivity analysis, I repeated the main analysis
  shown here separately for students in elementary (1st-5th), middle
  school (6th-8th), and high school (9th-12th) grades. These results are
  presented in the supplementary materials. Although more statistically
  noisy due to smaller sample sizes, those results are consistent with
  the conclusions drawn here.}

My goal is to understand the risk of being CBEG for each biracial group
relative to its constituent monoracial groups (e.g.~a Black/White
student compared to White and Black students). Throughout this article,
I use a visual approach to illustrate this placement. As an example, I
display the percent of White, Black, and Black/White students who are
CBEG in Figure~\ref{fig-example}. I display confidence bands/bars around
all three point estimates, but these confidence ranges require some
explanation as they are not standard 95\% confidence intervals. My goal
is to determine whether the biracial group's point estimate is
statistically distinguishable from a given monoracial constituent
group's point estimate by non-overlap of confidence intervals. Overlap
in 95\% confidence intervals is not equivalent to failing a hypothesis
test of difference at \(p<0.05\) and when used in this way, such
assessments lead to far more stringent tests \autocite{knol_mis_2011}.
When standard errors between groups are equal, overlap in 83.4\%
confidence intervals will achieve this goal. However, in cases where
standard errors vary substantially across estimates, the calculation of
the appropriate confidence interval is more complex and varies by each
pairwise comparison.\footnote{The formula for calculating the z-score
  for the correct interval is given by
  \[1.96 * \frac{\sqrt{1+\rho^2}}{1+\rho}\] where \(\rho\) is the ratio
  of the standard errors for the two statistics. See
  \textcite{knol_mis_2011} for a detailed derivation.}

\begin{figure}[t]

{\centering \includegraphics{main_files/figure-pdf/fig-example-1.pdf}

}

\caption{\label{fig-example}Probability of being clearly behind expected
grade for biracial black/white respondents in comparison to their
monoracial comparison groups. Non-overlap in color corresponding
confidence bands indicates statistically significant difference at the
5\% level.}

\end{figure}

I calculate two confidence intervals for each biracial group. Each
confidence interval is in comparison to one of the constituent
monoracial groups. These confidence intervals are color-coded in
Figure~\ref{fig-example} to indicate the reference group. I only draw
half-intervals in the direction of the monoracial group's point
estimate. Overlap in these color corresponding bars and bands indicates
that the difference between the biracial group and the monoracial group
is not statistically significant at \(p<.05\). For example, the yellow
bar shown in Figure~\ref{fig-example} for biracial Black/White students
does not quite overlap with the yellow band for White students,
indicating that the point estimates for these two groups are
statistically distinguishable at \(p<.05\).

Additionally, I include a measure of the ``halfway'' point between the
two monoracial groups in dark grey.\footnote{The halfway point is given
  by taking the mean between the two monoracial group. Its standard
  error is given by \(\sqrt{(s_1^2+s_2^2)/2}\) where \(s_1\) and \(s_2\)
  are the standard errors of the estimates for the two monoracial
  groups. To avoid clutter, I do not draw a third half-confidence
  interval for this point. Its length will be roughly halfway between
  the length of the other two intervals.} This halfway point allows me
to determine where the biracial group falls relative to the expectation
of being halfway between the two constituent monoracial groups. In this
case, Black/White students are much closer to White students in their
risk of grade retention, and have probabilities much lower than both
Black students and the halfway expectation.

Figure~\ref{fig-example} shows the raw differences between the three
racial groups of interest. However, in practice, I want to estimate
differences across groups while holding constant a variety of variables.
To do this, I estimate a set of logit models that predict the likelihood
of a student being CBEG by race and a variety of other variables. I then
construct figures similar to Figure~\ref{fig-example} by calculating
from the model the average predicted probabilities (APP) for each racial
group. An APP estimates the average probability of the outcome by a
given covariate across all cases, while holding constant all other
variables. APPs are akin to average marginal effects (AME) and estimated
in the same way. Whereas AMEs estimate differences or slopes, APPs
estimate the level for a given category.

I begin with a baseline model that adjusts for a variety of nuisance
characteristics that may vary across groups and thus need to be
controlled in all models. First, I include fixed effects for state of
residence because states dictate educational policy and thus can differ
substantially in the likelihood of grade retention. I also include dummy
variables indicating whether the student lived in a central city,
suburban, or rural area.\footnote{The location cannot be determined for
  all students and I therefore also include a fourth category of unknown
  in all models.}

The probability of being CBEG also increases with the student's current
grade, so I include fixed effects for the current grade of the student.
Additionally, as Figure~\ref{fig-year-grade} shows, the percent of
students who are CBEG has declined over time, but this decline has been
much more substantial at higher grade levels. For the elementary grades,
there is no evidence of a decline at all. To account for this
grade-specific decline in the models I include an interaction between a
linear year term and current grade. Sensitivity analysis showed that
this functional form was preferred by the Bayesian information criterion
to models with no interaction and a model with interaction terms that
treated year as a categorical variable. I also considered a similar
interaction between state and year, but this model was not
preferred.\footnote{The supplementary materials provide, for each model
  specification, the BIC scores and marginal probabilities of being
  clearly behind expected grade for each racial group. The marginal
  probabilities for each racial group are almost identical across
  specifications, indicating that results are not driven by the modeling
  choice made here.}

\begin{figure}[t]

{\centering \includegraphics{main_files/figure-pdf/fig-year-grade-1.pdf}

}

\caption{\label{fig-year-grade}Trends over time and grade in the percent
of students clearly behind expected grade. Lines are fit to each set of
points by grade via lowess smoothing.}

\end{figure}

The baseline model also includes the race of the respondent. I then add
additional terms measuring material and cultural resources that may
account for racial differences in a set of subsequent models. First, I
include measures of nativity and English proficiency for both the
respondent and each of their parents. Second, I include a categorical
measure of highest degree earned for each parent. Finally, I include
measures of family income (square rooted), home ownership, and whether
the parents are married.

All statistical analysis adjusts for sampling weights among respondents.
All models incorporate design effects for variance in sample weights and
the clustering of multiple respondents within the same household.

\hypertarget{results}{%
\section{Results}\label{results}}

\hypertarget{monoracial-differences}{%
\subsection{Monoracial Differences}\label{monoracial-differences}}

I begin by showing the relative risk of being CBEG among monoracial
respondents across models. Understanding these differences helps to
clarify the potential for how biracial respondents might be positioned
between monoracial groups. Table 2 reports average marginal effects on
the probability of being CBEG for each monoracial minority group in
comparison to White students.

\hypertarget{tbl-monoracial}{}
\begin{table}[t]
\caption{\label{tbl-monoracial}Average marginal differences in the probability of being clearly behind
expected grade by monoracial group, relative to a monoracial white
student }
\begin{center}
\begin{tabular}{l c c c c}
\hline
 & Model 1 & Model 2 & Model 3 & Model 4 \\
\hline
Black                  & $0.015^{***}$  & $0.016^{***}$  & $0.010^{***}$  & $0.007^{***}$  \\
                       & $(0.001)$      & $(0.001)$      & $(0.001)$      & $(0.001)$      \\
Indigenous             & $0.039^{***}$  & $0.039^{***}$  & $0.026^{***}$  & $0.022^{***}$  \\
                       & $(0.003)$      & $(0.003)$      & $(0.003)$      & $(0.003)$      \\
Asian                  & $-0.003^{***}$ & $-0.006^{***}$ & $-0.005^{***}$ & $-0.005^{***}$ \\
                       & $(0.001)$      & $(0.001)$      & $(0.001)$      & $(0.001)$      \\
Latino                 & $0.022^{***}$  & $0.013^{***}$  & $-0.001$       & $-0.002^{**}$  \\
                       & $(0.001)$      & $(0.001)$      & $(0.001)$      & $(0.001)$      \\
\hline
State fixed effects    & Yes            & Yes            & Yes            & Yes            \\
Location fixed effects & Yes            & Yes            & Yes            & Yes            \\
Grade fixed effects    & Yes            & Yes            & Yes            & Yes            \\
Year linear effects    & Yes            & Yes            & Yes            & Yes            \\
Year x grade effects   & Yes            & Yes            & Yes            & Yes            \\
Nativity and language  & No             & Yes            & Yes            & Yes            \\
Parent's education     & No             & No             & Yes            & Yes            \\
Other family resources & No             & No             & No             & Yes            \\
N                      & $2603331$      & $2603331$      & $2603331$      & $2603331$      \\
\hline
\multicolumn{5}{l}{\scriptsize{$^{***}p<0.001$; $^{**}p<0.01$; $^{*}p<0.05$}}
\end{tabular}
\label{tbl-monoracial}
\end{center}
\end{table}

Model 1 only controls for structural factors such as state of residence,
student location, grade, and year. These results provide a baseline
estimate of the differences across monoracial groups without controlling
for differences in cultural and material resources. The results show
that Black, Indigenous, and Latino students all have substantially
higher probability of being CBEG than White students. Indigenous
students have substantially higher risk than all other students, with a
probability of being CBEG that is 4.1\% percentage points higher than
White students. Asian students have the lowest probability of being
CBEG, and their risk is slightly lower than White students.

The subsequent models control for a variety of additional variables.
Model 2 includes controls for whether the student and each of their
parents are native-born and speak English well. Model 3 controls for the
highest degree received by each parent, and Model 4 controls for family
income (square rooted), home ownership, and whether the student's
parents are married.

Model 2 and Model 3 have substantial effects on the observed inequality
across racial groups, while the additional variables from Model 4 have
less impact. Controlling for the characteristics related to immigration
and acculturation in Model 2 cuts the gap between White and Latino
students in half and increases the gap between Asian and White students.
It has no impact on the gap between White and Black or White and
Indigenous students because most of these students are native-born with
native-born parents.

Controlling for education in Model 3 reduces the gap substantially for
Black and Indigenous students and removes the gap entirely for Latino
students. In fact, the results of both Models 3 and 4 suggest that, when
holding constant cultural and family resources, Latino students are
slightly less likely than White students to be CBEG.

In total, resource differentials account for a substantial part of the
overall racial differences in the risk of being CBEG, but not its
entireity. Black and Indigenous students remain at higher risk than
other students. Asian students remain the group with lowest risk across
all models. The most notable shift across models is for Latino students.
Their higher risk of being CBEG relative to White students is completely
accounted for by resource differentials. When comparing White and Latino
students with the same level or resources in Model 4, Latino students
actually have slightly lower risk of being CBEG.

Although, the racial differences in risk of CBEG are substantially
reduced by controlling for resource differentials, the remaining gaps
across all pairwise combinations are still large enough in most cases to
sustain the question of where biracial individuals fit within these
gaps. Before turning to this question, however, I want to better
understand how the resources of biracial individuals compare to their
monoracial constituent groups.

\hypertarget{the-distribution-of-resources-for-biracial-individuals}{%
\subsection{The Distribution of Resources for Biracial
Individuals}\label{the-distribution-of-resources-for-biracial-individuals}}

Table~\ref{tbl-resources} shows the mean values of the resource
variables across monoracial and biracial groups.\footnote{For
  compactness, I combine the parent foreign-born and English proficiency
  questions into whether either parent is foreign-born or speaks
  English. In the models, I use separate variables for mothers and
  fathers. Tables in the supplementary materials provide the full
  breakdown of these variables by racial group.} I also calculate a
counterfactual probability of CBEG for each racial group based solely on
their observed distribution of resources. Differences in this
counterfactual probability provide a summary measure of the resource
differentials between groups. This counterfactual probability is
calculated by estimating the average predicted probability for each
group from Model 4 of Table 2 when all non-resource variables, including
race, are held at their mean.\footnote{It might seem odd to hold
  categorical variables at their mean, but because all categorical
  variables are entered into the models as 0/1 numeric indicator
  variables, their mean value can be calculated as the proportion of
  cases with the indicated value.}

\hypertarget{tbl-resources}{}
\begin{table}
\caption{\label{tbl-resources}Mean resources by racial group. All results are survey weighted. Results
sorted by counterfactual probability of being clearly behind expected
grade. Shading indicates biracial group. }\tabularnewline

\centering
\resizebox{\linewidth}{!}{
\begin{tabular}[t]{l>{\raggedleft\arraybackslash}p{1.5cm}>{\raggedleft\arraybackslash}p{1.5cm}>{\raggedleft\arraybackslash}p{1.5cm}>{\raggedleft\arraybackslash}p{1.5cm}>{\raggedleft\arraybackslash}p{1.5cm}>{\raggedleft\arraybackslash}p{1.5cm}>{\raggedleft\arraybackslash}p{1.5cm}>{\raggedleft\arraybackslash}p{1.5cm}>{\raggedleft\arraybackslash}p{1.5cm}}
\toprule
Race & CBEG, counterfactual & Family income & Own home & Mother, 4-year college degree & Father, 4-year college degree & Foreign-born & Either parent foreign-born & Speak English well & Both parents speak english well\\
\midrule
\cellcolor[HTML]{D3D3D3}{White/Asian} & \cellcolor[HTML]{D3D3D3}{2.4\%} & \cellcolor[HTML]{D3D3D3}{\$187,049} & \cellcolor[HTML]{D3D3D3}{85.2\%} & \cellcolor[HTML]{D3D3D3}{63.9\%} & \cellcolor[HTML]{D3D3D3}{65.5\%} & \cellcolor[HTML]{D3D3D3}{6.2\%} & \cellcolor[HTML]{D3D3D3}{73.3\%} & \cellcolor[HTML]{D3D3D3}{99.4\%} & \cellcolor[HTML]{D3D3D3}{98.8\%}\\
\cellcolor[HTML]{D3D3D3}{Black/Asian} & \cellcolor[HTML]{D3D3D3}{2.9\%} & \cellcolor[HTML]{D3D3D3}{\$131,156} & \cellcolor[HTML]{D3D3D3}{64.6\%} & \cellcolor[HTML]{D3D3D3}{44.3\%} & \cellcolor[HTML]{D3D3D3}{39.4\%} & \cellcolor[HTML]{D3D3D3}{7.3\%} & \cellcolor[HTML]{D3D3D3}{81.7\%} & \cellcolor[HTML]{D3D3D3}{99.4\%} & \cellcolor[HTML]{D3D3D3}{98.6\%}\\
\cellcolor[HTML]{D3D3D3}{Latino/Asian} & \cellcolor[HTML]{D3D3D3}{2.9\%} & \cellcolor[HTML]{D3D3D3}{\$130,330} & \cellcolor[HTML]{D3D3D3}{70.3\%} & \cellcolor[HTML]{D3D3D3}{43.4\%} & \cellcolor[HTML]{D3D3D3}{40.5\%} & \cellcolor[HTML]{D3D3D3}{4.5\%} & \cellcolor[HTML]{D3D3D3}{71.1\%} & \cellcolor[HTML]{D3D3D3}{99.5\%} & \cellcolor[HTML]{D3D3D3}{96.8\%}\\
White & 3.1\% & \$136,953 & 83.3\% & 46.7\% & 43.6\% & 1.9\% & 9.4\% & 99.6\% & 98.9\%\\
Asian & 3.1\% & \$141,982 & 72.4\% & 56.7\% & 59.6\% & 21.9\% & 96.2\% & 96.5\% & 76.8\%\\
\cellcolor[HTML]{D3D3D3}{White/Latino} & \cellcolor[HTML]{D3D3D3}{3.3\%} & \cellcolor[HTML]{D3D3D3}{\$125,805} & \cellcolor[HTML]{D3D3D3}{73.0\%} & \cellcolor[HTML]{D3D3D3}{38.8\%} & \cellcolor[HTML]{D3D3D3}{36.1\%} & \cellcolor[HTML]{D3D3D3}{2.0\%} & \cellcolor[HTML]{D3D3D3}{32.9\%} & \cellcolor[HTML]{D3D3D3}{99.3\%} & \cellcolor[HTML]{D3D3D3}{98.5\%}\\
\cellcolor[HTML]{D3D3D3}{White/Indigenous} & \cellcolor[HTML]{D3D3D3}{3.6\%} & \cellcolor[HTML]{D3D3D3}{\$97,664} & \cellcolor[HTML]{D3D3D3}{70.7\%} & \cellcolor[HTML]{D3D3D3}{28.2\%} & \cellcolor[HTML]{D3D3D3}{24.2\%} & \cellcolor[HTML]{D3D3D3}{1.1\%} & \cellcolor[HTML]{D3D3D3}{9.6\%} & \cellcolor[HTML]{D3D3D3}{99.8\%} & \cellcolor[HTML]{D3D3D3}{99.9\%}\\
\cellcolor[HTML]{D3D3D3}{Indigenous/Asian} & \cellcolor[HTML]{D3D3D3}{3.6\%} & \cellcolor[HTML]{D3D3D3}{\$100,783} & \cellcolor[HTML]{D3D3D3}{56.1\%} & \cellcolor[HTML]{D3D3D3}{28.8\%} & \cellcolor[HTML]{D3D3D3}{22.6\%} & \cellcolor[HTML]{D3D3D3}{9.9\%} & \cellcolor[HTML]{D3D3D3}{71.3\%} & \cellcolor[HTML]{D3D3D3}{100.0\%} & \cellcolor[HTML]{D3D3D3}{97.0\%}\\
\cellcolor[HTML]{D3D3D3}{Black/White} & \cellcolor[HTML]{D3D3D3}{3.7\%} & \cellcolor[HTML]{D3D3D3}{\$94,841} & \cellcolor[HTML]{D3D3D3}{57.6\%} & \cellcolor[HTML]{D3D3D3}{31.9\%} & \cellcolor[HTML]{D3D3D3}{27.1\%} & \cellcolor[HTML]{D3D3D3}{2.0\%} & \cellcolor[HTML]{D3D3D3}{17.0\%} & \cellcolor[HTML]{D3D3D3}{99.8\%} & \cellcolor[HTML]{D3D3D3}{99.7\%}\\
Black & 3.8\% & \$87,159 & 53.9\% & 30.9\% & 25.1\% & 6.4\% & 27.7\% & 99.4\% & 96.2\%\\
\cellcolor[HTML]{D3D3D3}{Black/Latino} & \cellcolor[HTML]{D3D3D3}{3.8\%} & \cellcolor[HTML]{D3D3D3}{\$86,825} & \cellcolor[HTML]{D3D3D3}{47.2\%} & \cellcolor[HTML]{D3D3D3}{25.8\%} & \cellcolor[HTML]{D3D3D3}{23.4\%} & \cellcolor[HTML]{D3D3D3}{2.0\%} & \cellcolor[HTML]{D3D3D3}{32.4\%} & \cellcolor[HTML]{D3D3D3}{98.9\%} & \cellcolor[HTML]{D3D3D3}{98.5\%}\\
\cellcolor[HTML]{D3D3D3}{Black/Indigenous} & \cellcolor[HTML]{D3D3D3}{4.1\%} & \cellcolor[HTML]{D3D3D3}{\$76,097} & \cellcolor[HTML]{D3D3D3}{44.7\%} & \cellcolor[HTML]{D3D3D3}{19.3\%} & \cellcolor[HTML]{D3D3D3}{21.5\%} & \cellcolor[HTML]{D3D3D3}{3.9\%} & \cellcolor[HTML]{D3D3D3}{23.6\%} & \cellcolor[HTML]{D3D3D3}{100.0\%} & \cellcolor[HTML]{D3D3D3}{99.9\%}\\
\cellcolor[HTML]{D3D3D3}{Indigenous/Latino} & \cellcolor[HTML]{D3D3D3}{4.4\%} & \cellcolor[HTML]{D3D3D3}{\$72,164} & \cellcolor[HTML]{D3D3D3}{47.8\%} & \cellcolor[HTML]{D3D3D3}{14.4\%} & \cellcolor[HTML]{D3D3D3}{10.0\%} & \cellcolor[HTML]{D3D3D3}{1.1\%} & \cellcolor[HTML]{D3D3D3}{29.9\%} & \cellcolor[HTML]{D3D3D3}{99.3\%} & \cellcolor[HTML]{D3D3D3}{97.5\%}\\
Indigenous & 4.5\% & \$63,115 & 50.7\% & 13.2\% & 8.8\% & 7.1\% & 23.6\% & 99.2\% & 96.1\%\\
Latino & 5.3\% & \$62,226 & 50.2\% & 10.4\% & 8.5\% & 11.6\% & 83.0\% & 96.4\% & 50.8\%\\
\bottomrule
\end{tabular}}
\end{table}

The results show different risks of being CBEG across biracial groups
because of differences in resources. For example, White/Asian students
have the lowest counterfactual risk of being CBEG of all groups at
2.4\%, owing to their exceptionally high family income and the high
college attainment of their parents. At the other end of the spectrum,
Indigenous/Latino students have a counterfactual risk of CBEG nearly 2\%
higher, given their much lower family resources.

I am particularly interested in how each biracial group compares to its
constituent monoracial groups. For example, while Indigenous/Latino
students have the highest counterfactual risk of being CBEG among
biracial students, their risk is still lower than their two constituent
groups of Indigenous (4.5\%) and Latino (5.3\%) students, owing in large
part to somewhat higher family income, greater parental educational
attainment, and greater acculturation. These results suggest positive
selection into Indigenous/Latino interracial unions.

\hypertarget{tbl-resource-summary}{}
\begin{table}
\caption{\label{tbl-resource-summary}Relative placement for biracial groups in comparison to monoracial
constituent groups based on counterfactual probability of being clearly
behind expected grade due to resources alone. }\tabularnewline

\centering
\begin{tabular}[t]{>{\raggedright\arraybackslash}p{2cm}>{\raggedleft\arraybackslash}p{2cm}>{\raggedleft\arraybackslash}p{2cm}>{\raggedleft\arraybackslash}p{2cm}>{\raggedleft\arraybackslash}p{2cm}>{\raggedleft\arraybackslash}p{2cm}}
\toprule
Biacial & Lower probability than both & Similar probability as lower & Between & Similar probability as higher & Higher probability than both\\
\midrule
White/Asian & X &  &  &  & \\
Black/Asian &  & X (Asian) &  &  & \\
Latino/Asian &  & X (Asian) &  &  & \\
White/Latino &  & X (White) &  &  & \\
Black/Latino &  & X (Black) &  &  & \\
Indigenous/Latino &  & X (Indig.) &  &  & \\
White/Indigenous &  &  & X &  & \\
Black/Indigenous &  &  & X &  & \\
Indigenous/Latino &  &  & X &  & \\
Black/White &  &  &  & X (Black) & \\
\bottomrule
\end{tabular}
\end{table}

Table 4 summarizes the patterns from Table~\ref{tbl-resources} across
all ten biracial groups. In six of these cases, the biracial group would
have outcomes similar to or better than the monoracial group with better
resources. These results imply that for the majority of these groups,
parents in these matches are being positively selected from their
constituent groups.

Only three groups exhibit an in-between status based on resources. All
three of these groups are part-Indigenous. This finding does not hold
for Indigenous/Latino students, but nonetheless suggests less
selectivity in crossing this boundary.

The Black/White case stands out as an outlier. Black/White students have
similar resources as Black students and both groups have fewer resources
than White students, as can be seen in Table~\ref{tbl-resources} for
family income, home ownership, and parental educational attainment.
Unlike most of these other groups, I observe no positive class
selectivity into Black/White interracial unions.

The effect of controlling for these resources will be complex and
different across biracial groups. For those biracial groups with
resources similar to their more advantaged constituent group,
controlling for these resources should reduce their advantage relative
to the less advantaged monoracial group. For the Black/White case, on
the other hand, controlling for resource differentials should eliminate
some of their disadvantage relative to Whites. For the remaining
``in-between'' groups, it is more difficult to make a priori
predictions.

\hypertarget{biracial-placement}{%
\subsection{Biracial Placement}\label{biracial-placement}}

I now turn to the risk of CBEG for biracial respondents. Due to the
number of comparisons being made, I split these results into two
separate figures. In Figure~\ref{fig-black-models}, I show the placement
of each biracial group involving one black parent in comparison to their
two monoracial constituent groups. To allow comparison across models, I
compare the results from the baseline model (Model 1 in Table 2) and the
full model that accounts for all cultural and family resources (Model 4
in Table 2).

\begin{figure}[p]

{\centering \includegraphics{main_files/figure-pdf/fig-black-models-1.pdf}

}

\caption{\label{fig-black-models}Probability of being clearly behind
expected grade for biracial respondents with one black parent, in
comparison to their monoracial comparison groups. Non-overlap in color
corresponding confidence bands indicates statistically significant
difference at the 5\% level, Baseline model includes year, grade,
location, and state fixed effects. Full models include control variables
for nativity, English proficiency, income, education, home ownership,
and marital status of parents.}

\end{figure}

I focus first on biracial groups with one Black parent because of a
stronger historical expectation that these students will be identified
as Black by the ``one-drop rule'' and as a result will have similar
outcomes to Black students. Figure~\ref{fig-black-models} provides
little evidence for such a pattern, and soundly rejects it in the case
of Black/White and Black/Latino students. Black/White and Black/Latino
students have a risk of being CBEG substantially lower than Black
students and closer to their non-Black constituent group. Solid
conclusions are difficult for the remaining two groups because of
smaller sample size and correspondingly wider confidence intervals on
estimates. Nonetheless, the point estimates in each of these cases
suggest probabilities of being CBEG roughly halfway between the two
constituent groups.

Controlling for resource variables has some effect on the placement of
part-Black biracial respondents. Black/White biracial students initially
have a risk of being CBEG slightly higher than and statistically
distinguishable from White students. After controlling for cultural and
family resources, however, Black/White students have a risk of being
CBEG that is slightly lower but not statistically distinguishable from
White students. This change reflects the relatively low level of
parental education and income among Black/White students, which is much
closer to Black students than White students, as shown on
Table~\ref{tbl-resources}.

Black/Latino students start from a very different position, having a
risk of being CBEG substantially lower than both Black and Latino
students, who have somewhat similar risks. However, after controlling
for cultural and family resources, Black/Latino students have a risk of
being CBEG that is statistically indistinguishable from the risk of
Latino students and substantially lower than and statistically
distinguishable from Black students. This change largely reflects the
fact that Black/Latino students have greater resources than Latino (but
not Black) students and once this advantage is held constant, they have
a similar risk. After controlling for family resources, both Black/White
and Black/Latino students have a risk of being CBEG similar to their
monoracial constituent group with lower risk.

The role of parental resources is harder to determine for Black/Asian
and Black/Indigenous students due to the greater uncertainty in these
estimates. Controlling for resources moves the point estimate for
Black/Asian students from a risk similar to Asian students to more of a
halfway position between the two constituent groups. Black/Indigenous
students have risk of being CBEG roughly halfway between their two
constituent groups in both models. However, this finding of a halfway
position for both groups is highly tentative owing to wide confidence
intervals.

\begin{figure}[p]

{\centering \includegraphics{main_files/figure-pdf/fig-other-models-1.pdf}

}

\caption{\label{fig-other-models}Probability of being clearly behind
expected grade for non-black biracial respondents, in comparison to
their monoracial comparison groups. Non-overlap in color corresponding
confidence bands indicates statistically significant difference at the
5\% level, Baseline model includes year, grade, location, and state
fixed effects. Full models include control variables for nativity,
English proficiency, income, education, home ownership, and marital
status of parents.}

\end{figure}

Figure~\ref{fig-other-models} shows the results for the remaining
biracial groups. For most of these groups, the full models indicate that
each biracial group has a risk of being CBEG similar to the risk of the
lower monoracial constituent group or has risk higher than this group
but less than the halfway point. There are two exceptions to this
general pattern. First, the point estimate of the probability of being
CBEG for the Indigenous/Latino group is roughly halfway between the two
constituent groups, although it is not statistically distinguishable
from the lower group. Second, Latino/Asian students are the only case
where the point estimate of the risk for the biracial group is higher
than either monoracial group. However, the confidence bands suggest that
the risk for this group cannot be statistically differentiated from
either of the two monoracial constituent groups. This result only
emerges in the model that controls for resources, because Latino/Asian
students come from more advantaged households than both Latino and Asian
students, on average.

Aside from the case of Latino/Asian students, controlling for resources
has moderate effects on the placement of the remaining biracial groups.
White/Asian students are the only case with a substantial shift, owing
to their highly advantaged households. Initially White/Asian students
have risk of being CBEG substantially below both White and Asian
students. After controlling for resources, the risk of being CBEG for
White/Asian students is indistinguishable from those for Asian students,
but still substantially and statistically distinguishable from the
higher risk to White students.

Results for part-Latino students are complicated by the substantial
change in the monoracial Latino risk across models. For example, while
Latino students have substantially higher risk of being CBEG than White
students in the baseline model, there is only a very small difference
favoring Latino students in the full model. Nonetheless, even in this
case, we see that White/Latino students have risk closer to those of the
lower Latino group, but distinguishable from both groups.

\hypertarget{conclusions}{%
\section{Conclusions}\label{conclusions}}

In this article, I have used the risk of grade retention among K-12
students to better understand how biracial students' life chances
compare to their monoracial peers. Unlike prior work on this topic, the
estimates used here for most biracial groups are relatively precise due
to large sample sizes. For groups with a high level of precision, the
results tell a consistent story - biracial students' risk of grade
retention is similar to their monoracial constituent group with the
lower risk.

These results contradict expectations of both a ``halfway'' position and
the dominance of the one drop rule. For no group do I observe clear
evidence of a pattern of hypodescent in which the risk was similar to
the higher risk constituent monoracial group. These results also
strongly contradict the expectations of the marginal man hypothesis.

Two possibilities emerge that may help us understand these results
better. First, we may be witnessing a shift toward a new regime of
hyperdescent that applies broadly across a wide variety of racial
groups. Second, the relatively low risk for biracial students may be a
product of the strength of hybridity in allowing biracial students to
better negotiate racialized systems than their monoracial minority
peers.

Differences in family resources were expected to play some role in the
relative position of biracial students' risk. The results on parent
selection vary somewhat by group, but the majority of biracial groups
had resources more similar to their monoracial constituent group with
higher resources, indicating a strong positive selection into
interracial unions in the prior generation. The most important exception
to this trend is the case of Black/White biracial students who have far
fewer resources than White students and resources closer to Black
students.

Regardless of variation in biracial students' parental resources,
accounting for these resource differentials does not substantially
change the overall result that biracial students' risk is more similar
to the constituent monoracial group with lower risk. The one exception
to this finding is for the Latino/Asian case where the much lower risk
of these students compared to Latino and Asian students was entirely
driven by resource differentials.

Low relative risk for biracial individuals is sometimes treated as an
indication of a positive future direction for the US in terms of
ameliorating racial inequality. However, the improved prospects for
biracial individuals do nothing to ameliorate the often strong and
persistent inequalities between monoracial groups. The growth and
relative success of biracial populations may isolate remaining members
of monoracial minority groups as much as it assimilates mixed race
individuals. The real question at stake is how existing divides and
identities may be restructured in the context of a growing biracial
population. This research suggests the potential for a growing divide
between these mixed race populations and the most disadvantaged
monoracial minority populations from which they derive at least part of
their ancestry.

These findings stand in contrast to earlier work which found more
evidence of an ``in-between'' status in other educational outcomes
\autocite{herman_blackwhiteother_2009,campbell_multiracial_2009}. I
raise two possible explanations for this discrepancy that point to
strengths and weaknesses of the current research and may help drive
future work on the topic. First, this discrepancy may be due to the
small sample sizes of multiracial respondents in prior work which led to
statistically imprecise estimates. Because the current findings rely
upon large samples, they present an important step forward in our
understanding of how biracial individuals will fit into America's system
of racial inequality. Nonetheless, sample sizes for some non-white
biracial respondents remain small, limiting our understanding of the
outcomes for those groups. In particular, results for non-white
part-Indigenous populations are quite imprecise. The results
counter-intuitively suggest that part-Indigenous populations are the
least likely to be in the position expected by hyperdescent, but due to
low sample sizes, that finding is highly tentative.

Second, The current findings may also diverge from the findings of prior
work because different outcomes were examined. While grade retention is
an important early life outcome, it is far from the only one that these
individuals will face in their life. Although, we often observe similar
racial inequalities across a variety of outcomes, knowing the result for
one outcome does not perfectly predict other outcomes. As biracial
individuals age and face later outcomes in the educational system, the
labor market, and elsewhere, their experiences may differ from what I
observe here. Thus, collecting more detailed information on these
outcomes in future studies remains critically important.



\printbibliography[title=References]



\processdelayedfloats\end{document}
